\chapter*{Summary}
The data used in this work are from researchers using scanning electron microscope images
to investigate the properties of oxidation and coating layers on various materials essential to
the nuclear industry. Accurately measuring the thickness of these layers is important for
material characterization. The current manual measurement process is time-consuming, and
automating it can significantly improve efficiency.
Many algorithms can assist with automation, varying widely in complexity and input
expectations. While many of these algorithms promise powerful and accurate results, real-
world scenarios require substantial groundwork before these tools can be effectively
deployed and automation established.
This work discusses the challenges and solutions encountered, beginning with the collection and
assessment of all available data. It then explores experiments with both conventional
computer vision and machine learning algorithms, evaluating their performance. The overall
parameters of different solutions are examined, and the most suitable approach is selected.
This approach is then seamlessly integrated into the researchers' existing workflow.
\newpage
\chapter*{Súhrn}
Dáta použité v tejto práci pochádzajú od výskumníkov, ktorí využívajú snímky zo skenovacieho elektrónového mikroskopu na skúmanie vlastností oxidačných a povlakových vrstiev na rôznych materiáloch dôležitých pre jadrový priemysel. Presné meranie hrúbky týchto vrstiev je kľúčové pre charakterizáciu materiálov. Súčasný manuálny proces merania je časovo náročný a jeho automatizácia môže výrazne zvýšiť efektivitu.

Existuje mnoho algoritmov, ktoré môžu pomôcť s automatizáciou, pričom sa líšia svojou zložitosťou a požiadavkami na vstupné dáta. Hoci mnohé z nich ponúkajú presné a výkonné výsledky, v realite si ich nasadenie vyžaduje rozsiahlu prípravu, aby bolo možné efektívne dosiahnuť automatizáciu.

Táto práca sa zaoberá výzvami a riešeniami, ktoré boli pri vývoji automatizovaného riešenia identifikované. Začína zberom a analýzou dostupných dát, následne sa venuje experimentom s konvenčnými algoritmami počítačového videnia a strojového učenia, pričom hodnotí ich výkonnosť. Sú zhodnotené parametre rôznych riešení a vybraný najvhodnejší prístup, ktorý je následne integrovaný do existujúceho pracovného postupu.
\newpage

\chapter*{Acknowledgements}

    
I would like to express my gratitude to my advisor, \textbf{Ing. Petr Čech, Ph.D.}, and my consultant, \textbf{Mgr. Jaroslav Knotek}, for their guidance and support throughout this work.
\\
The input data used was created with state support from the Technology Agency of the Czech Republic under the THÉTA Program as part of project No. TK04030082 and utilizing the CICRR infrastructure, which is financially supported by the Ministry of Education, Youth, and Sports – project LM2023041.

\newpage