\chapter*{Summary}
The data used in this work are from researchers using scanning electron microscope images to investigate the properties of oxidation and coating layers on various materials essential to the nuclear industry. Accurately measuring the thickness of these layers is important for material characterization. The current manual measurement process is time-consuming, and automating it can significantly improve efficiency. Many algorithms can assist with automation, varying widely in complexity and input expectations. While many of these algorithms promise powerful and accurate results, real-world scenarios require substantial groundwork before these tools can be effectively deployed and automation established. This work discusses the challenges and solutions encountered, beginning with the collection and assessment of all available data. It then explores experiments with both conventional computer vision and machine learning algorithms, evaluating their performance. The overall parameters of different solutions are examined, and the most suitable approach is selected. This approach is then seamlessly integrated into the researchers' existing workflow.
\newpage
\chapter*{Souhrn}
Data použitá v této práci pocházejí od výzkumníků, kteří využívají snímky ze skenovacího elektronového mikroskopu ke zkoumání vlastností oxidačních a povlakových vrstev na různých materiálech důležitých pro jaderný průmysl. Přesné měření tloušťky těchto vrstev je klíčové pro charakterizaci materiálů. Současný manuální proces měření je časově náročný a jeho automatizace může výrazně zvýšit efektivitu. Existuje mnoho algoritmů, které mohou pomoci s automatizací, přičemž se liší svojí složitostí a požadavky na vstupní data. Ačkoli mnoho z nich nabízí přesné a výkonné výsledky, v realitě si jejich nasazení vyžaduje rozsáhlou přípravu, aby bylo možné efektivně dosáhnout automatizace. Tato práce se zabývá výzvami a řešeními, které byly při vývoji automatizovaného řešení identifikovány. Začíná sběrem a analýzou dostupných dat, následně se věnuje experimentům s konvenčními algoritmy počítačového vidění a strojového učení, přičemž hodnotí jejich výkonnost. Jsou zhodnoceny parametry různých řešení a vybraný nejvhodnější přístup, který je následně integrován do stávajícího pracovního postupu.
\newpage



\chapter*{Acknowledgements}

    
I would like to express my gratitude to my advisor, \textbf{Ing. Petr Čech, Ph.D.}, and my consultant, \textbf{Mgr. Jaroslav Knotek}, for their guidance and support throughout this work.
\\
The input data used was created with state support from the Technology Agency of the Czech Republic under the THÉTA Program as part of project No. TK04030082 and utilizing the CICRR infrastructure, which is financially supported by the Ministry of Education, Youth, and Sports – project LM2023041.

\newpage